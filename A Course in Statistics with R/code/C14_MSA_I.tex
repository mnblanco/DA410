\documentclass[]{article}
\usepackage{lmodern}
\usepackage{amssymb,amsmath}
\usepackage{ifxetex,ifluatex}
\usepackage{fixltx2e} % provides \textsubscript
\ifnum 0\ifxetex 1\fi\ifluatex 1\fi=0 % if pdftex
  \usepackage[T1]{fontenc}
  \usepackage[utf8]{inputenc}
\else % if luatex or xelatex
  \ifxetex
    \usepackage{mathspec}
  \else
    \usepackage{fontspec}
  \fi
  \defaultfontfeatures{Ligatures=TeX,Scale=MatchLowercase}
\fi
% use upquote if available, for straight quotes in verbatim environments
\IfFileExists{upquote.sty}{\usepackage{upquote}}{}
% use microtype if available
\IfFileExists{microtype.sty}{%
\usepackage{microtype}
\UseMicrotypeSet[protrusion]{basicmath} % disable protrusion for tt fonts
}{}
\usepackage[margin=1in]{geometry}
\usepackage{hyperref}
\hypersetup{unicode=true,
            pdftitle={C14: Multivariate Statistical Analysis - I},
            pdfborder={0 0 0},
            breaklinks=true}
\urlstyle{same}  % don't use monospace font for urls
\usepackage{graphicx,grffile}
\makeatletter
\def\maxwidth{\ifdim\Gin@nat@width>\linewidth\linewidth\else\Gin@nat@width\fi}
\def\maxheight{\ifdim\Gin@nat@height>\textheight\textheight\else\Gin@nat@height\fi}
\makeatother
% Scale images if necessary, so that they will not overflow the page
% margins by default, and it is still possible to overwrite the defaults
% using explicit options in \includegraphics[width, height, ...]{}
\setkeys{Gin}{width=\maxwidth,height=\maxheight,keepaspectratio}
\IfFileExists{parskip.sty}{%
\usepackage{parskip}
}{% else
\setlength{\parindent}{0pt}
\setlength{\parskip}{6pt plus 2pt minus 1pt}
}
\setlength{\emergencystretch}{3em}  % prevent overfull lines
\providecommand{\tightlist}{%
  \setlength{\itemsep}{0pt}\setlength{\parskip}{0pt}}
\setcounter{secnumdepth}{0}
% Redefines (sub)paragraphs to behave more like sections
\ifx\paragraph\undefined\else
\let\oldparagraph\paragraph
\renewcommand{\paragraph}[1]{\oldparagraph{#1}\mbox{}}
\fi
\ifx\subparagraph\undefined\else
\let\oldsubparagraph\subparagraph
\renewcommand{\subparagraph}[1]{\oldsubparagraph{#1}\mbox{}}
\fi

%%% Use protect on footnotes to avoid problems with footnotes in titles
\let\rmarkdownfootnote\footnote%
\def\footnote{\protect\rmarkdownfootnote}

%%% Change title format to be more compact
\usepackage{titling}

% Create subtitle command for use in maketitle
\newcommand{\subtitle}[1]{
  \posttitle{
    \begin{center}\large#1\end{center}
    }
}

\setlength{\droptitle}{-2em}

  \title{C14: Multivariate Statistical Analysis - I}
    \pretitle{\vspace{\droptitle}\centering\huge}
  \posttitle{\par}
    \author{}
    \preauthor{}\postauthor{}
    \date{}
    \predate{}\postdate{}
  
\usepackage{booktabs}
\usepackage{longtable}
\usepackage{array}
\usepackage{multirow}
\usepackage{wrapfig}
\usepackage{float}
\usepackage{colortbl}
\usepackage{pdflscape}
\usepackage{tabu}
\usepackage{threeparttable}
\usepackage{threeparttablex}
\usepackage[normalem]{ulem}
\usepackage{makecell}
\usepackage{xcolor}

\begin{document}
\maketitle

\hypertarget{graphical-plots-for-multivariate-data}{%
\subsection{14.2 Graphical Plots for Multivariate
Data}\label{graphical-plots-for-multivariate-data}}

\hypertarget{example-14.2.1.-car-data.}{%
\subsubsection{EXAMPLE 14.2.1. Car
Data.}\label{example-14.2.1.-car-data.}}

\includegraphics{C14_MSA_I_files/figure-latex/unnamed-chunk-2-1.pdf}
\includegraphics{C14_MSA_I_files/figure-latex/unnamed-chunk-2-2.pdf}

\hypertarget{example-14.2.2.-car-data.}{%
\subsubsection{EXAMPLE 14.2.2. Car
Data.}\label{example-14.2.2.-car-data.}}

\begin{verbatim}
## [1] "Warning: NA elements have been exchanged by mean values!!"
\end{verbatim}

\includegraphics{C14_MSA_I_files/figure-latex/unnamed-chunk-3-1.pdf}

\begin{verbatim}
## effect of variables:
##  modified item       Var  
##  "height of face   " "P"  
##  "width of face    " "M"  
##  "structure of face" "R78"
##  "height of mouth  " "R77"
##  "width of mouth   " "H"  
##  "smiling          " "R"  
##  "height of eyes   " "Tr" 
##  "width of eyes    " "W"  
##  "height of hair   " "L"  
##  "width of hair   "  "T"  
##  "style of hair   "  "D"  
##  "height of nose  "  "G"  
##  "width of nose   "  "C"  
##  "width of ear    "  "P"  
##  "height of ear   "  "M"
\end{verbatim}

\hypertarget{definitions-notations-and-summary-statistics-for-multivariate-data}{%
\subsection{14.3 Definitions, Notations, and Summary Statistics for
Multivariate
Data}\label{definitions-notations-and-summary-statistics-for-multivariate-data}}

\hypertarget{example-14.3.1.-plot-of-a-few-bivariate-normal-densities.}{%
\subsubsection{EXAMPLE 14.3.1. Plot of a few Bivariate Normal
Densities.}\label{example-14.3.1.-plot-of-a-few-bivariate-normal-densities.}}

\includegraphics{C14_MSA_I_files/figure-latex/unnamed-chunk-4-1.pdf}

\hypertarget{example-14.3.2}{%
\subsubsection{EXAMPLE 14.3.2}\label{example-14.3.2}}

Normally Distributed and Uncorrelated Random Variables Do Not Imply that
the Random Variables are Independent.

\includegraphics{C14_MSA_I_files/figure-latex/unnamed-chunk-5-1.pdf}

A counter example of the myth that uncorrelated and normal distribution
imply independence

\hypertarget{example-14.3.3.-the-board-stiffness-dataset.}{%
\subsubsection{EXAMPLE 14.3.3. The Board Stiffness
Dataset.}\label{example-14.3.3.-the-board-stiffness-dataset.}}

\begin{verbatim}
## Warning in mean.default(stiff): argument is not numeric or logical:
## returning NA
\end{verbatim}

\begin{verbatim}
## [1] NA
\end{verbatim}

\begin{verbatim}
##           x1        x2       x3        x4
## x1 105616.30  94613.53 87289.71  94230.73
## x2  94613.53 101510.12 76137.10  81064.36
## x3  87289.71  76137.10 91917.09  90352.38
## x4  94230.73  81064.36 90352.38 104227.96
\end{verbatim}

\includegraphics{C14_MSA_I_files/figure-latex/unnamed-chunk-6-1.pdf}

\hypertarget{example-14.3.4.-conversion-of-a-variance-covariance-matrix-into-a-correlation-matrix.}{%
\subsubsection{EXAMPLE 14.3.4. Conversion of a Variance-Covariance
Matrix into a Correlation
Matrix.}\label{example-14.3.4.-conversion-of-a-variance-covariance-matrix-into-a-correlation-matrix.}}

\begin{verbatim}
##              Sepal.Length Sepal.Width Petal.Length Petal.Width
## Sepal.Length    1.0000000  -0.1174687    0.8717360   0.8179881
## Sepal.Width    -0.1174687   1.0000000   -0.4284721  -0.3659896
## Petal.Length    0.8717360  -0.4284721    1.0000000   0.9628602
## Petal.Width     0.8179881  -0.3659896    0.9628602   1.0000000
\end{verbatim}

\begin{verbatim}
##              Sepal.Length Sepal.Width Petal.Length Petal.Width
## Sepal.Length    1.0000000  -0.1174687    0.8717360   0.8179881
## Sepal.Width    -0.1174687   1.0000000   -0.4284721  -0.3659896
## Petal.Length    0.8717360  -0.4284721    1.0000000   0.9628602
## Petal.Width     0.8179881  -0.3659896    0.9628602   1.0000000
\end{verbatim}

\hypertarget{early-outlier-detection}{%
\subsection{14.3.1 Early Outlier
Detection}\label{early-outlier-detection}}

\hypertarget{example-14.3.5.-the-board-stiffness-dataset.-contd.}{%
\subsubsection{EXAMPLE 14.3.5. The Board Stiffness Dataset.
Contd.}\label{example-14.3.5.-the-board-stiffness-dataset.-contd.}}

\includegraphics{C14_MSA_I_files/figure-latex/unnamed-chunk-8-1.pdf}

\begin{verbatim}
##      x1   x2   x3   x4   x1   x2   x3   x4
## 1  1889 1651 1561 1778 -0.1 -0.3  0.2  0.2
## 2  2403 2048 2087 2197  1.5  0.9  1.9  1.5
## 3  2119 1700 1815 2222  0.7 -0.2  1.0  1.5
## 4  1645 1627 1110 1533 -0.8 -0.4 -1.3 -0.6
## 5  1976 1916 1614 1883  0.2  0.5  0.3  0.5
## 6  1712 1712 1439 1546 -0.6 -0.1 -0.2 -0.6
## 7  1943 1685 1271 1671  0.1 -0.2 -0.8 -0.2
## 8  2104 1820 1717 1874  0.6  0.2  0.7  0.5
## 9  2983 2794 2412 2581  3.3  3.3  3.0  2.7
## 10 1745 1600 1384 1508 -0.5 -0.5 -0.4 -0.7
## 11 1710 1591 1518 1667 -0.6 -0.5  0.0 -0.2
## 12 2046 1907 1627 1898  0.4  0.5  0.4  0.5
## 13 1840 1841 1595 1741 -0.2  0.3  0.3  0.0
## 14 1867 1685 1493 1678 -0.1 -0.2 -0.1 -0.1
## 15 1859 1649 1389 1714 -0.1 -0.3 -0.4  0.0
## 16 1954 2149 1180 1281  0.1  1.3 -1.1 -1.4
## 17 1325 1170 1002 1176 -1.8 -1.8 -1.7 -1.7
## 18 1419 1371 1252 1308 -1.5 -1.2 -0.8 -1.3
## 19 1828 1634 1602 1755 -0.2 -0.4  0.3  0.1
## 20 1725 1594 1313 1646 -0.6 -0.5 -0.6 -0.2
## 21 2276 2189 1547 2111  1.1  1.4  0.1  1.2
## 22 1899 1614 1422 1477  0.0 -0.4 -0.3 -0.8
## 23 1633 1513 1290 1516 -0.8 -0.7 -0.7 -0.6
## 24 2061 1867 1646 2037  0.5  0.4  0.5  1.0
## 25 1856 1493 1356 1533 -0.2 -0.8 -0.5 -0.6
## 26 1727 1412 1238 1469 -0.6 -1.1 -0.9 -0.8
## 27 2168 1896 1701 1834  0.8  0.5  0.6  0.3
## 28 1655 1675 1414 1597 -0.8 -0.2 -0.3 -0.4
## 29 2326 2301 2065 2234  1.3  1.7  1.8  1.6
## 30 1490 1382 1214 1284 -1.3 -1.2 -1.0 -1.4
\end{verbatim}

\begin{verbatim}
##  [1]  0.6000129  5.4770196  7.6166439  5.2076098  1.3980776  2.2191409
##  [7]  4.9883498  1.4876570 12.2647550  0.7665400  1.9307771  0.4635159
## [13]  2.6959024  0.1295714  1.0792484 16.8474070  3.5018290  3.9900603
## [19]  1.3632124  1.4649908  9.8980384  5.0557446  0.7962096  2.5385575
## [25]  4.5767867  3.3979804  2.3816050  2.9951752  6.2837628  2.5838186
\end{verbatim}

For each variable, there is one observation at the right-hand side of
the diagram whose value is very large compared with the rest of the
observations.

\hypertarget{testing-for-mean-vectors-one-sample}{%
\subsection{14.4 Testing for Mean Vectors : One
Sample}\label{testing-for-mean-vectors-one-sample}}

Random vector samples from a multivariate normal distribution.

\hypertarget{testing-h-m-m0-s-known}{%
\subsection{14.4.1 Testing H : m = m0, S
Known}\label{testing-h-m-m0-s-known}}

\hypertarget{example-14.4.1.-the-importance-of-handling-the-covariances.}{%
\subsubsection{EXAMPLE 14.4.1. The Importance of Handling the
Covariances.}\label{example-14.4.1.-the-importance-of-handling-the-covariances.}}

\begin{verbatim}
##        [,1]
## [1,] 8.4026
\end{verbatim}

\begin{verbatim}
## [1] 5.991465
\end{verbatim}

\begin{verbatim}
## [1] 1.45
\end{verbatim}

\begin{verbatim}
## [1] -0.7495332
\end{verbatim}

Since the calculated 𝜒2 value is greater than the tabulated value, we
reject the hypothesis that the average height and weight are at 70 and
170 units respectively.

The absolute value of each of these tests is less than 1.96, and hence
we would have failed to reject the hypothesis H, which is not the case
when the correlations are adjusted for. Thus, we learn an important
story that whenever the correlations are known, it is always better to
adjust the statistical procedure for them.

\hypertarget{example-14.4.2.-the-board-stiffness-data.-contd.}{%
\subsubsection{EXAMPLE 14.4.2. The Board Stiffness Data.
Contd.}\label{example-14.4.2.-the-board-stiffness-data.-contd.}}

\begin{verbatim}
##          [,1]
## [1,] 365.9636
\end{verbatim}

\begin{verbatim}
## [1] 9.487729
\end{verbatim}

The test procedure thus rejects the hypothesis H, since the value of the
test statistic is larger than the critical value.

\hypertarget{testing-h-m-m0-s-unknown}{%
\subsection{14.4.2 Testing H : m = m0, S
Unknown}\label{testing-h-m-m0-s-unknown}}

\hypertarget{example-14.4.3.-the-calcium-in-soil-and-turnip-greenes-data-of-rencher-2002.}{%
\subsubsection{EXAMPLE 14.4.3. The Calcium in Soil and Turnip Greenes
Data of Rencher
(2002).}\label{example-14.4.3.-the-calcium-in-soil-and-turnip-greenes-data-of-rencher-2002.}}

When the variance-covariance matrix is unknown.

\begin{enumerate}
\def\labelenumi{(\roman{enumi})}
\tightlist
\item
  x1: available calcium in the soil, (ii) x2: exchangeable soil calcium,
  and (iii) x3: turnip green calcium
\end{enumerate}

\begin{verbatim}
##          [,1]
## [1,] 24.55891
\end{verbatim}

\begin{verbatim}
## 
##  Hotelling's one sample T2-test
## 
## data:  calcium
## T.2 = 367.16, df1 = 3, df2 = 7, p-value = 4.648e-08
## alternative hypothesis: true location is not equal to c(0,0,0)
\end{verbatim}

\begin{verbatim}
## 
##  Hotelling's one sample T2-test
## 
## data:  calcium
## T.2 = 1416.2, df = 3, p-value < 2.2e-16
## alternative hypothesis: true location is not equal to c(0,0,0)
\end{verbatim}

\begin{table}[H]
\centering
\begin{tabular}{l|r}
\hline
  & x\\
\hline
y1 & 28.100\\
\hline
y2 & 7.180\\
\hline
y3 & 3.089\\
\hline
\end{tabular}
\end{table}

\hypertarget{example-14.4.4.-the-calcium-in-soil-and-turnip-greenes-data-of-rencher-2002.contd.}{%
\subsubsection{EXAMPLE 14.4.4. The Calcium in Soil and Turnip Greenes
Data of Rencher
(2002).Contd.}\label{example-14.4.4.-the-calcium-in-soil-and-turnip-greenes-data-of-rencher-2002.contd.}}

\begin{verbatim}
## 
##  Hotelling's one sample T2-test
## 
## data:  calcium
## T.2 = 6.3671, df1 = 3, df2 = 7, p-value = 0.02068
## alternative hypothesis: true location is not equal to c(15,6,2.85)
\end{verbatim}

The conclusion is to reject the hypothesis H.

\hypertarget{example-14.4.5.-the-calcium-in-soil-and-turnip-greenes-data-of-rencher-2002.contd.}{%
\subsubsection{EXAMPLE 14.4.5. The Calcium in Soil and Turnip Greenes
Data of Rencher
(2002).Contd.}\label{example-14.4.5.-the-calcium-in-soil-and-turnip-greenes-data-of-rencher-2002.contd.}}

\begin{verbatim}
## 
##  Hotelling's one sample T2-test
## 
## data:  calcium
## T.2 = 24.559, df = 3, p-value = 1.909e-05
## alternative hypothesis: true location is not equal to c(15,6,2.85)
\end{verbatim}

\hypertarget{testing-for-mean-vectors-two-samples}{%
\subsection{14.5 Testing for Mean Vectors :
Two-Samples}\label{testing-for-mean-vectors-two-samples}}

Random vector samples from two plausible populations.

\hypertarget{example-14.5.1.-psychological-tests-for-males-and-females.}{%
\subsubsection{EXAMPLE 14.5.1. Psychological Tests for Males and
Females.}\label{example-14.5.1.-psychological-tests-for-males-and-females.}}

The hypothesis of interest is to test H ∶ \(\mu_1 = \mu2\) against K:
\(\mu_1 \ne \mu2\)

Psychological Tests for Males and Females. A psychological study
consisting of four tests was conducted on male and female groups and the
results were noted. Since the four tests are correlated and each one is
noted for all the individuals, we are interested in knowing if the mean
vector of the test scores is the same across the gender group. The four
tests here are as follows:

\begin{itemize}
\tightlist
\item
  x1: pictorial inconsistencies
\item
  x2: paper form board
\item
  x3: tool recognition
\item
  x4: vocabulary.
\end{itemize}

Assumption: Assume that the covariance matrix is the same for both the
groups, and that it is unknown.

\begin{verbatim}
## [1] 32
\end{verbatim}

\begin{verbatim}
## [1] 32
\end{verbatim}

\begin{verbatim}
##     M_y1     M_y2     M_y3     M_y4 
## 15.96875 15.90625 27.18750 22.75000
\end{verbatim}

\begin{verbatim}
##     F_y1     F_y2     F_y3     F_y4 
## 12.34375 13.90625 16.65625 21.93750
\end{verbatim}

\begin{verbatim}
##          M_y1      M_y2      M_y3      M_y4
## M_y1 7.164315  6.047379  5.693044  4.700605
## M_y2 6.047379 15.894153  8.492440  5.855847
## M_y3 5.693044  8.492440 29.356351 13.980847
## M_y4 4.700605  5.855847 13.980847 22.320565
\end{verbatim}

\begin{verbatim}
##         [,1]
## [1,] 97.6015
\end{verbatim}

\begin{verbatim}
## 
##  Hotelling's two sample T2-test
## 
## data:  males and females
## T.2 = 23.22, df1 = 4, df2 = 59, p-value = 1.464e-11
## alternative hypothesis: true location difference is not equal to c(0,0,0,0)
\end{verbatim}

\begin{verbatim}
## 
##  Hotelling's two sample T2-test
## 
## data:  males and females
## T.2 = 97.601, df = 4, p-value < 2.2e-16
## alternative hypothesis: true location difference is not equal to c(0,0,0,0)
\end{verbatim}

Comparing the value of the test statistic \(T^2\) with the critical
value \(T^2_{0.01, 4, 62}\) = 15.373, we reject the hypothesis of equal
mean vectors for the gender groups.

\hypertarget{multivariate-analysis-of-variance}{%
\subsection{14.6 Multivariate Analysis of
Variance}\label{multivariate-analysis-of-variance}}

\hypertarget{wilks-test-statistic}{%
\subsection{14.6.1 Wilks Test Statistic}\label{wilks-test-statistic}}

\hypertarget{example-14.6.1.-apple-of-different-rootstock.}{%
\subsubsection{EXAMPLE 14.6.1. Apple of Different
Rootstock.}\label{example-14.6.1.-apple-of-different-rootstock.}}

Test if the mean vector of the four variables is the same across six
stratas of the experiment, that is, H ∶ \(\mu_1 = \mu2 = ... = \mu6\)

\begin{verbatim}
##         y1    y2    y3    y4
## [1,] 0.320  1.70 0.554 0.217
## [2,] 1.697 12.14 4.364 2.110
## [3,] 0.554  4.36 4.291 2.482
## [4,] 0.217  2.11 2.482 1.723
\end{verbatim}

\begin{verbatim}
##          y1    y2    y3    y4
## [1,] 0.0736 0.537 0.332 0.208
## [2,] 0.5374 4.200 2.355 1.637
## [3,] 0.3323 2.355 6.114 3.781
## [4,] 0.2085 1.637 3.781 2.493
\end{verbatim}

\begin{verbatim}
##         y1    y2     y3    y4
## [1,] 0.394  2.23  0.886 0.426
## [2,] 2.234 16.34  6.719 3.747
## [3,] 0.886  6.72 10.405 6.263
## [4,] 0.426  3.75  6.263 4.216
\end{verbatim}

\begin{verbatim}
## [1] 0.154
\end{verbatim}

\begin{verbatim}
## The following objects are masked _by_ .GlobalEnv:
## 
##     rootstock, y1, y2, y3
\end{verbatim}

\begin{verbatim}
##                     Df Wilks approx F num Df den Df Pr(>F)  
## rootstock$rootstock  1 0.764     3.32      4     43  0.019 *
## Residuals           46                                      
## ---
## Signif. codes:  0 '***' 0.001 '**' 0.01 '*' 0.05 '.' 0.1 ' ' 1
\end{verbatim}

The calculated values of Wilks lambda = 0.154 is less than the
theoretical value of 0.455 (corresponding to p = 4, \(_vH\) = 5,
\(_vE\)E = 42). Thus, we reject the hypothesis that the mean vector is
the same for the six strata.

\hypertarget{roys-test}{%
\subsection{14.6.2 Roy's Test}\label{roys-test}}

\begin{verbatim}
##                     Df   Roy approx F num Df den Df Pr(>F)  
## rootstock$rootstock  1 0.308     3.32      4     43  0.019 *
## Residuals           46                                      
## ---
## Signif. codes:  0 '***' 0.001 '**' 0.01 '*' 0.05 '.' 0.1 ' ' 1
\end{verbatim}

The Roy's test also rejects the hypothesis H that the mean vector for
the six strata are equal.

\hypertarget{pillai-test-statistic}{%
\subsection{14.6.3 Pillai Test Statistic}\label{pillai-test-statistic}}

\begin{verbatim}
##                     Df Pillai approx F num Df den Df Pr(>F)  
## rootstock$rootstock  1  0.236     3.32      4     43  0.019 *
## Residuals           46                                       
## ---
## Signif. codes:  0 '***' 0.001 '**' 0.01 '*' 0.05 '.' 0.1 ' ' 1
\end{verbatim}

The Pillai's test statistic confirms the findings of the Wilks's and
Roy's test that the mean vector for the six strata are significantly
different.

\hypertarget{the-lawleyhotelling-test-statistic}{%
\subsection{14.6.4 The Lawley'Hotelling Test
Statistic}\label{the-lawleyhotelling-test-statistic}}

\begin{verbatim}
##                     Df Hotelling-Lawley approx F num Df den Df Pr(>F)  
## rootstock$rootstock  1            0.308     3.32      4     43  0.019 *
## Residuals           46                                                 
## ---
## Signif. codes:  0 '***' 0.001 '**' 0.01 '*' 0.05 '.' 0.1 ' ' 1
\end{verbatim}

The Lawley'Hotelling's test also rejects the hypothesis H that the mean
vector for the six strata are equal.

It is concluded that all the four statistical tests lead to the same
conclusion, that the mean vector for the six strata are different.

\hypertarget{example-14.6.2.testing-for-physico-chemical-properties-of-water-in-4-cities.}{%
\subsubsection{EXAMPLE 14.6.2.Testing for Physico-chemical Properties of
Water in 4
Cities.}\label{example-14.6.2.testing-for-physico-chemical-properties-of-water-in-4-cities.}}

Test if the properties are the same across the four cities and in which
case a same water treatment approach can be adopted for all cities.

\begin{verbatim}
##           Df  Wilks approx F num Df den Df Pr(>F)    
## City       3 0.0305     12.8     27    150 <2e-16 ***
## Residuals 59                                         
## ---
## Signif. codes:  0 '***' 0.001 '**' 0.01 '*' 0.05 '.' 0.1 ' ' 1
\end{verbatim}

\begin{verbatim}
##           Df  Roy approx F num Df den Df Pr(>F)    
## City       3 6.31     37.2      9     53 <2e-16 ***
## Residuals 59                                       
## ---
## Signif. codes:  0 '***' 0.001 '**' 0.01 '*' 0.05 '.' 0.1 ' ' 1
\end{verbatim}

\begin{verbatim}
##           Df Pillai approx F num Df den Df Pr(>F)    
## City       3   1.91     10.3     27    159 <2e-16 ***
## Residuals 59                                         
## ---
## Signif. codes:  0 '***' 0.001 '**' 0.01 '*' 0.05 '.' 0.1 ' ' 1
\end{verbatim}

\begin{verbatim}
##           Df Hotelling-Lawley approx F num Df den Df Pr(>F)    
## City       3             8.59     15.8     27    149 <2e-16 ***
## Residuals 59                                                   
## ---
## Signif. codes:  0 '***' 0.001 '**' 0.01 '*' 0.05 '.' 0.1 ' ' 1
\end{verbatim}

The four statistical tests indicates that the mean vector of the ten
variates across the four cities are significantly different from each
other, the water treatment program across the cities has to be
different.

\hypertarget{testing-for-variance-covariance-matrix-one-sample}{%
\subsection{14.7 Testing for Variance-Covariance Matrix: One
Sample}\label{testing-for-variance-covariance-matrix-one-sample}}

\hypertarget{example-14.7.1.-understanding-the-height-weight-relationship.}{%
\subsubsection{EXAMPLE 14.7.1. Understanding the Height-Weight
Relationship.}\label{example-14.7.1.-understanding-the-height-weight-relationship.}}

In MSA, the covariance matrix plays the role of the scale parameter. We
thus naturally encounter the problem of testing H ∶ \(\sum = \sum_0\)s

\begin{verbatim}
## [1] 11.1
\end{verbatim}

\begin{verbatim}
## [1] 10.7
\end{verbatim}

\begin{verbatim}
## [1] 7.81
\end{verbatim}

Since the values of u and u′ are greater than the critical value, we
reject the hypothesis \$H∶𝚺=𝚺0 a ndconcludethat𝚺≠𝚺0◽

\hypertarget{testing-for-sphericity}{%
\subsection{14.7.1 Testing for
Sphericity}\label{testing-for-sphericity}}

\hypertarget{example-14.7.2.-the-linguistic-probe-word-analysis.}{%
\subsubsection{EXAMPLE 14.7.2. The Linguistic Probe Word
Analysis.}\label{example-14.7.2.-the-linguistic-probe-word-analysis.}}

The problem of testing if the components of a random vector are
independent is equivalent to the problem of testing H ∶ 𝚺 = 𝜎2I, where I
is the identity matrix.

The problem of a test for independence of the components is also known
as tests of sphericity.

Probe words are used to test the recall ability of words in various
linguistic contexts. In this experiment the response time to 5 different
probe words are recorded for 11 individuals. The interest in the
experiment is to examine if the response times to the different words
are independent or not. The failure to reject the hypothesis of
sphericity implies that the response times can be compared using ANOVA.

\begin{verbatim}
## [1] 0.0395
\end{verbatim}

\begin{verbatim}
## [1] 26.2
\end{verbatim}

\hypertarget{testing-for-variance-covariance-matrix-k-samples}{%
\subsection{14.8 Testing for Variance-Covariance Matrix:
k-Samples}\label{testing-for-variance-covariance-matrix-k-samples}}

\hypertarget{testing-for-equality-of-covariance-matrices}{%
\section{Testing for Equality of Covariance
Matrices}\label{testing-for-equality-of-covariance-matrices}}

\hypertarget{example-14.8.1.-psychological-tests-for-males-and-females.-contd.}{%
\subsubsection{EXAMPLE 14.8.1. Psychological Tests for Males and
Females.
Contd.}\label{example-14.8.1.-psychological-tests-for-males-and-females.-contd.}}

\begin{verbatim}
## [1] 14.6
\end{verbatim}

\hypertarget{the-boxs-chi-square-approximation}{%
\section{The Box's chi-square
approximation}\label{the-boxs-chi-square-approximation}}

\begin{verbatim}
## [1] 18.3
\end{verbatim}

\begin{verbatim}
## [1] 13.6
\end{verbatim}

\begin{verbatim}
## [1] 18.3
\end{verbatim}

\begin{verbatim}
## [1] 1.35
\end{verbatim}

\begin{verbatim}
## [1] 1.83
\end{verbatim}

\hypertarget{testing-for-independence-of-sub-vectors}{%
\subsection{14.9 Testing for Independence of
Sub-vectors}\label{testing-for-independence-of-sub-vectors}}

\hypertarget{example-14.9.1.-the-seishuwine-study.}{%
\subsubsection{EXAMPLE 14.9.1. The SeishuWine
Study.}\label{example-14.9.1.-the-seishuwine-study.}}

\hypertarget{sheishu---read.csvseishu_wine.csvheadertrue}{%
\section{sheishu \textless{}-
read.csv(``Seishu\_wine.csv'',header=TRUE)}\label{sheishu---read.csvseishu_wine.csvheadertrue}}

\begin{verbatim}
## [1] 0.0163
\end{verbatim}

\begin{verbatim}
## [1] 74
\end{verbatim}

\begin{verbatim}
## [1] 930
\end{verbatim}

\begin{verbatim}
## [1] 37
\end{verbatim}

\begin{verbatim}
## [1] 0.838
\end{verbatim}

\begin{verbatim}
## [1] 100
\end{verbatim}

\begin{verbatim}
## [1] 69.3
\end{verbatim}


\end{document}
